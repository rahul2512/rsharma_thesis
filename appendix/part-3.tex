\chapter{Mathematical detail}\label[appendix]{app3}


\section{Rotations in three-dimensions, $SO(3)$ group} \label{a3:s1}
The special orthogonal matrix group, $SO(3)$ represents all proper rotations in three-dimensional Euclidean space, i.e., $\in \R^3$ and is defined as:
\begin{equation}
SO(3) = \{ R \in \R^{3\times3} \;|\; R^{T}R = RR^{T} = I \in \R^{3\times3},\; \det{R}=+1 \}
\label{a3:eq1}    
\end{equation}
 where $I$ is an identity matrix. 
$R$ is a proper right-handed rotation matrix with 
\{$1,e^{\iota\theta},e^{-\iota\theta}$\} as eigenvalues, where $\theta$ is 
the angle of rotation, and the real eigenvector of $R$ is the axis of 
rotation $u$. 
    
Furthermore, for a given rotation matrix, $R$, the Euler-Rodrigues formula gives the following relation between the rotation matrix $R$ and a unit axis of rotation $u$ and angle of rotation $\theta$: 
\begin{equation}
SO(3) \ni R = \cos{\theta}I + (1-\cos{\theta})u\otimes{u} + \sin{\theta}u^{\times}
\label{a3:eq2}    
\end{equation}
where $I \in \R^{3\times3}$ is an identity matrix, $u\otimes{u} = uu^{T}$ is outer-product and $u^{\times}$ is a skew-symmetric matrix satisfying $(u^{\times})v = u\times{v}\; \forall \; v \in \R^3$ and of the form:
\begin{equation}
u^{\times} = \begin{bmatrix}
    0 & -u_{3} & u_2 \\
    u_3 & 0  & -u_1 \\
    -u_2 & u_1 & 0 \\
   \end{bmatrix}
\label{a3:eq3}    
\end{equation}

A skew-symmetric matrix $u^{\times}$ can be transformed to its corresponding as $u=\text{vec}(u^{\times})$. 

Using Euler-Rodrigues formula in \Cref{a3:eq2}, a direct relation between $R$ and  $u$, $\theta$ is given as:
\begin{equation}
[0,\pi) \ni \theta = \arccos\left( \frac{tr(R)-1}{2}\right) \;\; \text{and} \;\; \R^{3} \ni u = \frac{2}{1+tr(R)} \text{vec}(R-R^{T})
\label{a3:eq4}    
\end{equation}
Note that, for $\theta = 0 \; \text{and}\; \pi$, the rotation matrix become symmetric which means $R-R^T$ will be a zero-matrix and thus, can't be used for the computation of rotation axis, $u$. In the case of $\theta=0$, $Q$ becomes an identity matrix, and any unit vector can be the rotation axis. However, when $\theta=\pi$, the axis of rotation will be the eigenvector of matrix $Q+I$.  
%%%%%%%%%%%%%%%%%%%%%%%%%%%%%%%%%%%%%%%%%%%%%  %%%%%%%%%%%%%%%%%%%%%%%%%%%%%%%%%%%%%%%%%%%%%  %%%%%%%%%%%%%%%%%%%%%%%%%%%%%%%%%%%%%%%%%%%%%  %%%%%%%%%%%%%%%%%%%%%%%%%%%%%%%%%%%%%%%%%%%%%  %%%%%%%%%%%%%%%%%%%%%%%%%%%%%%%%%%%%%%%%%%%%%
\section{Parameterisation of rotations in cgDNA$+$ model}\label{a3:s2}
Now, to parameterise rotations in cgDNA$+$ model,
we have used Cayley parameters (details in ~\cite{lankavs2009parameterization,patelithesis,petthesis}). We have defined the function $cay: \R^3 \xrightarrow{} SO(3)$ as: 

\begin{equation}
cay_{\alpha}(\eta) = I + \frac{1}{4\alpha^{2}+|\eta|^{2}} \left[  4\alpha \eta^{\times} + 2(\eta^{\times})^2 \right] = R(u,\theta) \; \forall \; \alpha \in \R, \eta \in \R^{3}
\label{a3:eq5}    
\end{equation}

where $\R^3 \ni u=\frac{\eta}{|\eta|}$ and $\theta=2\arctan{\left(\frac{|\eta|}{2\alpha}\right)}$. 
The inverse of $cay$ transformation can be defined as $cay^{-1}$: $SO(3) \xrightarrow{} \R^3$ and is given in \Cref{a3:eq6}.

\begin{equation}
cay_{\alpha}^{-1}(R) = \frac{2\alpha}{1+tr(R)} \text{vec}(R-R^{T})
\label{a3:eq6}    
\end{equation}

\section{Rigid body transformation, $SE(3)$ group} \label{a3:s3}
To describe the position and orientation of rigid body, we have used 
special euclidean group, $SE(3)$ which is defined as: 
\begin{equation}
SE(3) = \Big\{  \R^{4\times4} \ni G =  
\begin{bmatrix}
    R & r  \\
    0 & 1   \\
   \end{bmatrix} \Big\}
\label{a3:eq7}    
\end{equation}
where $R \in SO(3)$ is the rotational component and $ r \in \R^3$ is the translational component of the rigid body transformation. 

The product of $G_1 \in SE(3)$ and $G_2 \in SE(3)$ is given as 
\begin{equation}
 SE(3) \ni G_1G_2 =  
\begin{bmatrix}
    R_1R_2 & R_1r_2 +r_1 \\
    0 & 1   \\
   \end{bmatrix}
\label{a3:eq8}    
\end{equation}
and the inverse of an element in $SE(3)$ is, 
\begin{equation}
 SE(3) \ni G^{-1} =  
\begin{bmatrix}
    R^T & -R^Tr \\
    0 & 1   \\
   \end{bmatrix}
\label{a3:eq9}    
\end{equation}

%%%%%%%%%%%%%%%%%%%%%%%%%%%%%%%%%%%%%%%%%%%%%%%%%%%%%%%%%%%%%%%%%%%%%%%%%%%%%%%%%%%%%%%%%%%%%%%%%%%%%%%%%%%%%%%%%%%%%%%%%%%%%%%%%%%%%%%%%%%%%%%%%%%%%%%%%%%%%%%%%%%%%%%%%%%%%%%%%%%%%%%%%%%%%%%%%%%%%%%%%%%%%%%%%%%%%%%%%%%%%%%%%%%%%%%%%%%%%%%%%%%%%%%%%%%%%%%%%%%%%%%%%%%%%%%%%%%%%%%%%%%%%%%%%%%%%%%%%%%%%%%%%%%%%%%%%%%
\section{Kullback-Leibler divergence} \label{a3:s4}
The Kullback-Leibler (KL) divergence~\cite{kld}, also known as relative entropy, between two continuous 
pdfs $\rho_1(x) \; \text{and} \; \rho_2(x)$ defined on $\Omega \subset \R^N$ is given as: 
\begin{equation}
D_{KL}(\rho_1(x),\rho_2(x)) = \int_{\Omega}\rho_1(x)\log\frac{\rho_1(x)}{\rho_2(x)}dx \geq 0  
\label{a3:eq11}    
\end{equation}
where the equality holds if $\rho_1=\rho_2$. 

Some of the key properties of KL divergence are: 
\begin{itemize}
    \item KL divergence is non-symmetric, i.e., $D_{KL}(\rho_1,\rho_2) \ne D_{KL}(\rho_2,\rho_1)$, in general. 
\item KL divergence doesn't satisfy triangle inequality and doesn't qualify as a metric. It just defines a premetric on the set of pdfs. 
\item Invariant under re-scaling i.e say $X_1,X_2$ are random 
variables associated to pdfs $\rho_1, \rho_2$ and $X'_1,X'_2$ are random 
variables associated to pdfs $\rho'_1, \rho'_2$ where $\{X_i = aX'_i\}_{i=1,2}$, then $D_{KL}(\rho_1,\rho_2)=D_{KL}(\rho'_1,\rho'_2)$.
This invariance of KL divergence under re-scaling allowed an easier 
re-scaling of rotational coordinates in cgDNA family of models and change of reading strand. 
\item In case, when the pdfs $\rho_1, \rho_2$ are normal multivariate distributions, \cref{a3:eq11} simplifies to an algebraic form, 
\begin{equation}
\begin{split}
D_{KL}(\rho_1,\rho_2)  &=\S(\rho_1,\rho_2) + \M(\rho_1,\rho_2) \\
\S(\rho_1,\rho_2) &= \frac{1}{2}\left[ K_1^{-1}:K_2
-\ln\left(\frac{|K_2|}{|K_1|}\right)-I:I\right] \\  
\M(\rho_1,\rho_2)&= \frac{1}{2} (\mu_1-\mu_2)^T K_2(\mu_1-\mu_2)
 \end{split}
\label{a3:eq12}  
\end{equation}
where $\mu_1$ and $\mu_2$ are mean vectors, $K_1$ and $K_2$ are inverse
covariance matrices, and $:$ represents the standard Euclidean inner product 
for square-matrices and $I$ is the identity matrix of the size same as $K_1$ and $K_2$. $\sqrt{\M}$ is also known as Mahalanobis distance~\cite{mahal}.  
\end{itemize}

Moreover, KL divergence can be symmetrised as follows:
\begin{equation}
\begin{split}
D_{KLS}(\rho_1,\rho_2) &= \frac{1}{2}[D_{KL}(\rho_1,\rho_2) + D_{KL}(\rho_2,\rho_1)]\\
 & =  \S_{S}(\rho_1,\rho_2) + \M_{S}(\rho_1,\rho_2),\\
\S_{S}(\rho_1,\rho_2) & =  \frac{1}{2}\left[ K_1^{-1}:K_2 + K_2^{-1}:K_1 -2I:I\right],\\
\M_{S}(\rho_1,\rho_2)&=  \frac{1}{2} \left[(\mu_1-\mu_2)^T (K_2+K_1) (\mu_1-\mu_2)\right],
\end{split}
\label{a3:eq13}
\end{equation}
where $\S_{S}(\rho_1,\rho_2)$ is symmetrised stiffness contribution,
and $\M_{S}(\rho_1,\rho_2)$ is symmetrised shape contribution of the symmetrised KL divergence. 

\clearpage