\chapter{An involution of $3\times 3$ block structure}\label[appendix]{app4}


Let $P$ be an orthogonal matrix
\begin{equation}
P = \begin{bmatrix}
0 & 0 & P_1\\
0 & P_2 & 0\\
P_1 & 0 & 0
\end{bmatrix}
\text{ where } P^2 = I, P^TP = I, P \in n \times n,  P_{i}^2 = I, P_i \in n_i \times n_i \text{ and } P_{i}^T = P_{i} \; \forall \; i = {1,2}
\end{equation}
and a symmetric matrix $K$ 
\begin{equation}
K =  
\begin{bmatrix}
A & B & E\\
B^T & C & D\\
E^T & D^T & F
\end{bmatrix}
\text{ such that } K^T=K
\end{equation}
which is also an involution symmetric matrix such that $K = PKP = P^TKP$. 
Let's define another orthogonal matrix $Q$, 
\begin{equation}
Q = \frac{1}{\sqrt{2}} 
\begin{bmatrix}
I & 0 & I\\
0 & \sqrt{2}I & 0\\
-P_1 & 0 & P_1
\end{bmatrix}
\text{ such that } Q^TQ=I
\end{equation}

Now, if $P_2$ is diagonal matrix with $s$ elements -1 and rest of the $r$ elements +1 (as $\pm1$ are the only possibility as $P_2^2=I$ )
then the involution symmetry of $K$ implies that the orthogonal transformation $Q^TKQ$ yields a $2\times2$ block structure, i.e.,
\begin{equation}
Q^TKQ = 
\begin{bmatrix}
H_1 & O \\
O & H_2 
\end{bmatrix}
\text{ where } H_1 \in (n_1+s) \times (n_1+s)
\text{ and } H_2 \in (n_1+r) \times (n_1+r)
\end{equation}
\begin{equation}
PKP = 
\begin{bmatrix}
P_1FP_1 & P_1D^TP_2 & P_1E^TP_1\\
P_2DP_1 & P_2CP_2 & P_2BP_1\\
P_1EP_1 & P_1B^TP_2 & P_1AP_1
\end{bmatrix}
\end{equation}
so the symmetry and involution symmetry iff 4 conditions (independent) satisfy
\begin{itemize}
    \item $A = P_1FP_1$
  \item $P_1B= D^TP_2$
  \item $EP_1 = P_1E^T=(EP_1)^T$
  \item $P_2CP_2=C$
\end{itemize}

Now, 
\mathleft
\begin{equation}
\begin{split}
Q^TKQ  = & 
\frac{1}{2}
\begin{bmatrix}
I & 0 & -P_1\\
0 & \sqrt{2}I & 0\\
I & 0 & P_1
\end{bmatrix}
\begin{bmatrix}
A & B & E\\
B^T & C & D\\
E^T & D^T & F
\end{bmatrix}
\begin{bmatrix}
I & 0 & I\\
0 & \sqrt{2}I & 0\\
-P_1 & 0 & P_1
\end{bmatrix} \\
 = & 
 \frac{1}{2}
\begin{bmatrix}
I & 0 & -P_1\\
0 & \sqrt{2}I & 0\\
I & 0 & P_1
\end{bmatrix}
\begin{bmatrix}
A-EP_1 & \sqrt{2}B & A+EP_1\\
B^T -DP_1 & \sqrt{2}C & B^T+DP_1\\
 E^T-FP_1 & \sqrt{2}D^T & E^T+FP_1
\end{bmatrix}\\
 = & 
 \frac{1}{2}
\begin{bmatrix}
A+P_1FP_1-P_1E^T-EP_1 & \sqrt{2}[B-P_1D^T] & A-P_1FP_1+EP_1-P_1E^T\\
\sqrt{2}[B^T -DP_1] & 2C & \sqrt{2}[B^T+DP_1]\\
 A-P_1FP_1-P_1E^T-EP_1 & \sqrt{2}[B+P_1D^T] & A+P_1FP_1+EP_1+P_1E^T
\end{bmatrix}\\
= & \ \ \ 
\begin{bmatrix}
A-EP_1 & \frac{1}{\sqrt{2}}B[I-P_2] & O\\
\frac{1}{\sqrt{2}}[I-P_2]B^T & C & \frac{1}{\sqrt{2}}[I+P_2]B^T)\\
 O & \frac{1}{\sqrt{2}}B[I+P_2] & A+EP_1
\end{bmatrix}\\
 = & 
 \frac{1}{2}
\begin{bmatrix}
H_1 & O \\
O & H_2
\end{bmatrix}\\
\end{split}    
\end{equation}
\mathcenter
\text{where }
$H_1 :=$ \begin{bmatrix}
A-EP_1 & \sqrt{2}B_1 \\
\sqrt{2}B_1^T & C_{11}
\end{bmatrix}
 \text{ and }
$H_2 := $ \begin{bmatrix}
C_{22} & \sqrt{2}B_2 \\
\sqrt{2}B_2^T & A+EP_1
\end{bmatrix}.

Lemma
\begin{itemize}
    \item Orthogonal similarity doesn't change eigenvalues and systems for $H_1$ and $H_2$ are decoupled, i.e.,  $\lambda(K) = \mu(H_1) \cup \gamma(H_2)$
    \item If $(\mu_i,\begin{bmatrix}
    \underbar{w}_i \\
    \underbar{x}_i
    \end{bmatrix})$ is eigenpair for $H_1$ then corresponding eigenpair for $K$ is $(\mu_i,Q \begin{bmatrix}
    \underbar{w}_i \\
    \underbar{x}_i\\
    o 
    \end{bmatrix} = \frac{1}{\sqrt{2}}
    \begin{bmatrix}
        \underbar{w}_i \\
    \sqrt{2}\hat{\underbar{x}}_i \\
    -P_1\underbar{w}_i
    \end{bmatrix}
    )$ where $\hat{\underbar{x}}_i = \begin{bmatrix}
        \underbar{x}_i\\
        0
    \end{bmatrix}$
    \item Similarly, if $(\gamma_j,\begin{bmatrix}
    \underbar{y}_j \\
    \underbar{z}_j
    \end{bmatrix})$ is eigenpair for $H_2$ then corresponding eigenpair for $K$ is $(\gamma_j,Q \begin{bmatrix}
    o \\
    \underbar{y}_j \\
    \underbar{z}_j
    \end{bmatrix} = \frac{1}{\sqrt{2}}
    \begin{bmatrix}
        \underbar{z}_j \\
    \sqrt{2}\hat{\underbar{y}}_j \\
    +P_1\underbar{z}_j
    \end{bmatrix}
    ) 
    $ where $\hat{\underbar{y}}_j = \begin{bmatrix}
        0 \\
        \underbar{y}_j
        \end{bmatrix}$
\end{itemize}

It explains the sparsity pattern in eigenvectors of a palindromically symmetrized matrix.
\clearpage