%\addcontentsline{toc}{section}{\bf Résumé}
{\centerline {{\textbf{Résumé}}}}
\vspace{0.75cm}
La mécanique de l'ADN joue un rôle crucial dans de nombreux processus biologiques, notamment le positionnement des nucléosomes et les interactions protéine-ADN.
On pense que la nature utilise des modifications épigénétiques de l'ADN pour réguler davantage l'expression des gènes.
De plus, l'ARN double brin et l'hybride ADN:ARN (DRH) sont également importants en biologie, et leur mécanique joue un rôle important.
Il est maintenant bien établi que la mécanique d'un acide nucléique double brin (ADNd) est fonction de sa séquence.
En particulier, la mécanique de l'ADN séquence-dépendant est souvent considérée comme le \say{code génétique secondaire} en raison de son rôle primordial dans la lecture de l'ADN.
Cependant, une compréhension complète de la mécanique séquence-dépendant des acides nucléiques (AN) fait toujours défaut, principalement en raison de l'énorme espace des séquences, qui est inexplorable en utilisant des expériences ou des simulations atomistiques de dynamique moléculaire (MD), et nécessite donc une alternative précise et efficace.


Cette thèse étend le modèle cgDNA$+$, un modèle à gros grains séquence-dépendant de l'ADNdb, à cgNA$+$ en estimant les paramètres pour divers ADNdb, y compris l'ARN, le DRH et l'ADN avec des modifications épigénétiques.
Le modèle est entraîné sur des simulations MD atomistiques en utilisant les protocoles MD les plus récents.
Pour une séquence arbitraire, le modèle prédit efficacement les distributions d'équilibre séquence-dépendant, en traitant les bases et les phosphates comme des corps rigides.
Le modèle est évalué de manière approfondie pour des séquences de test mécaniquement diverses et divers choix de modélisation sont expliqués et justifiés en quantifiant l'erreur associée.

De plus, ayant d'abord montré dans les données de structure aux rayons X protéine-ADN que les contextes flanquants sont essentiels pour la mécanique des dimères, nous avons comparé les observations aux rayons X aux prédictions du modèle pour les dimères dans tous les contextes de tétramères et avons trouvé un accord raisonnable pour la forme moyenne, la rigidité, la direction de variation de l'état fondamental dans l'espace de séquence, et la direction de la déformation de l'ADN dans l'espace de configuration.
De manière remarquable, nous avons également trouvé un excellent alignement entre la direction de variation de l'état fondamental dans l'espace des séquences et la direction de la déformation de l'ADN dans l'espace de configuration, ce qui implique que, pour diverses séquences/contextes de flanquement, le dimère adopte l'état fondamental en faisant plus de compromis dans les modes souples de l'espace de configuration.

L'efficacité du modèle permet d'étudier des propriétés intéressantes des AN, telles que la forme moyenne, la longueur de persistance, les conformations du  chaîne principale et la largeur des sillons pour des millions de séquences, ce qui permet de tirer des conclusions statistiques sur l'espace des séquences.
Elle permet d'aborder des questions telles que (a) quel polymorphisme nucléotidique unique influence le moins/le plus la mécanique de l'ADN et sa sensibilité à la séquence flanquante, (b) le rôle de la séquence dans le rétrécissement/l'élargissement des sillons, et (c) le rôle de la séquence flanquante dans les modifications épigénétiques.
D'autres applications comprennent le scanning des génomes à la recherche de séquences mécaniquement exceptionnelles, la compréhension de l'enroulement/déroulement des nucléosomes séquence-dépendant, la prédiction de l'affinité de liaison des protéines et l'étude de la réponse de AN aux charges externes.

Enfin, nous développons un outil d'apprentissage profond pour prédire l'emplacement des atomes de sucre dans toute configuration à gros grains d'ADNc$+$. Il permet de générer un ensemble de configurations atomistiques pour toute séquence comparable aux simulations MD mais en un temps très court et d'étudier les conformations du  chaîne principale et des sucres. De plus, une structure d'équilibre à grain fin séquence-dépendant peut être utilisée pour démarrer les simulations MD, ce qui est particulièrement utile pour les mini-cercles d'ADN.


% \vskip 0.5cm
% \noindent
% \textbf{Mots-clés:} {\it Grainage grossier}, {\it Simulations MD}, {\it Mécanique de l'ADN}, {\it ARN}, {\it Hybride ADN:ARN}, {\it Épigénétique}, {\it Réseau neuronal}, {\it Friction du sucre}, {\it Largeur des sillons}, {\it Longueur de persistance}.


%, {\it B\rom{1}-B\rom{2} DNA backbone conformations}.


%The findings indicate that the change in groundstate of DNA on epigenetic modifications is significantly larger when \cpg step is present in the \cpg islands.


%%gene regulation and genome formation of many viruses, and their structure and flexibility are pivotal in their functioning.
